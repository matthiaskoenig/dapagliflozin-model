\begin{landscape}
\begin{table}[H]
\centering
\tabcolsep=3pt
\renewcommand{\arraystretch}{1.3}
\tiny
\begin{threeparttable} 

\caption{\scriptsize{\textbf{Summary of published computational models for dapagliflozin.} 
Overview of published computational models including model type, software/platform, 
reproducibility criteria (open software, open model, open code, open data, 
reproducibility, FAIR, longterm storage), resources, clinical data sources, and model scope.}}
\label{tab:computational_models_overview}

\begin{tabularx}{\linewidth}{
    >{\raggedright\arraybackslash}p{1.6cm}  % Study
    >{\centering\arraybackslash}p{0.8cm}    % PubMed ID
    >{\raggedright\arraybackslash}p{1.0cm}  % Model Type
    >{\raggedright\arraybackslash}p{1.6cm}  % Platform/Software
    >{\centering\arraybackslash}p{0.8cm}    % Open Software
    >{\centering\arraybackslash}p{0.8cm}    % Open Model
    >{\centering\arraybackslash}p{0.8cm}    % Open Code
    >{\centering\arraybackslash}p{0.8cm}    % Open Data
    >{\centering\arraybackslash}p{0.8cm}    % Reproducibility
    >{\centering\arraybackslash}p{0.8cm}    % FAIR
    >{\centering\arraybackslash}p{0.8cm}    % Longterm Storage
    >{\centering\arraybackslash}p{2.3cm}    % Resources
    >{\centering\arraybackslash}p{0.7cm}    % Studies
    >{\raggedright\arraybackslash}p{3.0cm}  % Clinical Data Used
    >{\raggedright\arraybackslash}p{5.0cm}  % Scope
}
\toprule
\textbf{Study} & \textbf{PubMed ID} & \textbf{Model Type} & \textbf{Platform/Software} & \textbf{Open Software} & \textbf{Open Model} & \textbf{Open Code} & \textbf{Open Data} & \textbf{Reproducibility} & \textbf{FAIR} & \textbf{Longterm Storage} & \textbf{Resources} & \textbf{Studies} & \textbf{Clinical Data Used} & \textbf{Scope} \\
\midrule
Balazki2018 \cite{Balazki2018} & \href{https://pubmed.ncbi.nlm.nih.gov/30270578/}{30270578} & QSP + PBPK/PD & PK-Sim, MoBi, OSPS, R & \cellcolor{green!20}Yes & \cellcolor{green!20}Yes & \cellcolor{green!20}Yes & \cellcolor{green!20}Yes & \cellcolor{red!20}No & \cellcolor{red!20}No & \cellcolor{red!20}No & https://github.com/Open-Systems-Pharmacology/SGLT2i-hyperfiltration-model Model, Data, Code in supplement; & 7 & 7 clinical studies; digitized literature data & Renal hyperfiltration, tubuloglomerular feedback, glucose-sodium reabsorption, SGLT2 inhibition effects on GFR \\
\addlinespace[1pt]
Busse2019 \cite{Busse2019} & \href{https://pubmed.ncbi.nlm.nih.gov/31077437/}{31077437} & PopPK + PK/PD & NONMEM, PsN, R & \cellcolor{red!20}No & \cellcolor{red!20}No & \cellcolor{red!20}No & \cellcolor{red!20}No & \cellcolor{red!20}No & \cellcolor{red!20}No & \cellcolor{red!20}No & - & 2 & 2 clinical studies (T1DM adolescents + adults; PK + UGE ER); clinical studies referenced & Pharmacokinetics, exposure-urinary glucose excretion relationship, body weight and eGFR effects in Type 1 diabetes (adolescents vs adults) \\
\addlinespace[1pt]
Callegari2021 \cite{Callegari2021} & \href{https://pubmed.ncbi.nlm.nih.gov/33314761/}{33314761} & PBPK & Simcyp & \cellcolor{red!20}No & \cellcolor{red!20}No & \cellcolor{red!20}No & \cellcolor{red!20}No & \cellcolor{red!20}No & \cellcolor{red!20}No & \cellcolor{red!20}No & - & 1 & 1 clinical DDI study (ertugliflozin + mefenamic acid), literature PK for dapagliflozin; & PBPK modeling (ertugliflozin), UGT-mediated drug-drug interaction prediction with mefenamic acid, absorption and metabolism pathways \\
\addlinespace[1pt]
Guo2025 \cite{Guo2025} & \href{https://pubmed.ncbi.nlm.nih.gov/40230691/}{40230691} & PBPK/PD & PK-Sim, MoBi, OriginLab & \cellcolor{green!20}Yes & \cellcolor{red!20}No & \cellcolor{red!20}No & \cellcolor{red!20}No & \cellcolor{red!20}No & \cellcolor{red!20}No & \cellcolor{red!20}No & - & 25 & ~25 clinical PK/PD studies & PBPK/PD modeling of four SGLT2 inhibitors (dapagliflozin, canagliflozin, empagliflozin, ipragliflozin), enal tubule structure, renal glucose reabsorption, urinary glucose excretion, dose optimization in T2DM with renal insufficiency \\
\addlinespace[1pt]
Jo2021 \cite{Jo2021} & \href{https://pubmed.ncbi.nlm.nih.gov/33439535/}{33439535} & PBPK & Simcyp & \cellcolor{red!20}No & \cellcolor{red!20}No & \cellcolor{red!20}No & \cellcolor{red!20}No & \cellcolor{red!20}No & \cellcolor{red!20}No & \cellcolor{red!20}No & - & 7 & 7 clinical studies; trials registered on ClinicalTrials.gov & PBPK model incorporating UGT1A9 ontogeny for pediatric dose selection, DDIs with rifampin/mefenamic acid, hepatic/renal impairment predictions \\
\addlinespace[1pt]
Maurer2011 \cite{Maurer2011} & \href{https://pubmed.ncbi.nlm.nih.gov/21870203/}{21870203} & PK/PD & NONMEM & \cellcolor{red!20}No & \cellcolor{red!20}No & \cellcolor{red!20}No & \cellcolor{red!20}No & \cellcolor{red!20}No & \cellcolor{red!20}No & \cellcolor{red!20}No & - & 1 & 1 published study: Komoroski et al. 2009; digitized literature data; data not shared & Biologically-based PK/PD model of UGE, rat-to-human translational pharmacology \\
\addlinespace[1pt]
Mori2016 \cite{Mori2016} & \href{https://pubmed.ncbi.nlm.nih.gov/27604638/}{27604638} & PBPK/PD & Simcyp, simBio & \cellcolor{red!20}No & \cellcolor{red!20}No & \cellcolor{red!20}No & \cellcolor{red!20}No & \cellcolor{red!20}No & \cellcolor{red!20}No & \cellcolor{red!20}No & - & 14 & 14 studies total: 8 canagliflozin studies, 6 dapagliflozin studies; clinical studies referenced & PBPK/PD model predicting canagliflozin and dapagliflozin concentrations in intestinal lumen and renal proximal tubules, SGLT1/2 inhibition ratios, urinary glucose excretion validation \\
\addlinespace[1pt]
Shah2021 \cite{Shah2021} & \href{https://pubmed.ncbi.nlm.nih.gov/33368935/}{33368935} & QSP PK/PD & Monolix, R & \cellcolor{red!20}No & \cellcolor{red!20}No & \cellcolor{red!20}No & \cellcolor{red!20}No & \cellcolor{red!20}No & \cellcolor{red!20}No & \cellcolor{red!20}No & - & 5 & 5 studies total (1 phase IIa + 4 phase III), trials registered on ClinicalTrials.gov & Quantitative systems pharmacology (QSP) model integrating dapagliflozin PK, glucose-insulin homeostasis, renal glucose reabsorption, and HbA1c formation to predict treatment effect in T2DM patients \\
\addlinespace[1pt]
Shahidehpour2024 \cite{Shahidehpour2024} & \href{https://pubmed.ncbi.nlm.nih.gov/39160349/}{39160349} & Mechanistic PK & Python & \cellcolor{green!20}Yes & \cellcolor{red!20}No & \cellcolor{red!20}No & \cellcolor{red!20}No & \cellcolor{red!20}No & \cellcolor{red!20}No & \cellcolor{red!20}No & - & - & Secondary literature values (number of studies not specified) & Methodology for estimating drug clearance in chronic kidney disease (CKD) using probability density functions from secondary data, mechanistic models, and PK first principles�applied to metformin and dapagliflozin PK modeling and dose adjustment \\
\addlinespace[1pt]
Sokolov2019 \cite{Sokolov2019} & \href{https://pubmed.ncbi.nlm.nih.gov/30456904/}{30456904} & PK/PD & NONMEM, R & \cellcolor{red!20}No & \cellcolor{red!20}No & \cellcolor{red!20}No & \cellcolor{red!20}No & \cellcolor{red!20}No & \cellcolor{red!20}No & \cellcolor{red!20}No & - & 2 & 2 clinical studies (not publicly available) & Exposure-response modeling of dapagliflozin PK and 24h-UGE in T1DM patients, comparing Japanese vs non-Japanese populations with covariate effects (eGFR, SMBG, insulin dose) \\
\addlinespace[1pt]
vanderAart\-vanderBeek2021 \cite{vanderAartvanderBeek2021} & \href{https://pubmed.ncbi.nlm.nih.gov/33587286/}{33587286} & PopPK & NONMEM, R & \cellcolor{red!20}No & \cellcolor{red!20}No & \cellcolor{red!20}No & \cellcolor{red!20}No & \cellcolor{red!20}No & \cellcolor{red!20}No & \cellcolor{red!20}No & - & 1 & 1 clinical trial & Population PK modeling in non-diabetic CKD patients and linking exposure to changes in kidney risk markers \\
\addlinespace[1pt]
VanDerWalt2013 \cite{VanDerWalt2013} & \href{https://pubmed.ncbi.nlm.nih.gov/23887724/}{23887724} & PopPK & NONMEM, R & \cellcolor{red!20}No & \cellcolor{green!20}Yes & \cellcolor{green!20}Yes & \cellcolor{red!20}No & \cellcolor{red!20}No & \cellcolor{red!20}No & \cellcolor{red!20}No & NONMEM code in supplement & 3 & 3 clinical studies & Population PK model for dapagliflozin and its inactive metabolite D3OG, quantifying renal vs hepatic contributions to UGT1A9-mediated metabolism in subjects with renal/hepatic impairment \\
\addlinespace[1pt]
Yao2023 \cite{Yao2023} & \href{https://pubmed.ncbi.nlm.nih.gov/36890732/}{36890732} & Meta-analytic PK/PD & NONMEM, R & \cellcolor{red!20}No & \cellcolor{red!20}No & \cellcolor{red!20}No & \cellcolor{red!20}No & \cellcolor{red!20}No & \cellcolor{red!20}No & \cellcolor{red!20}No & - & 80 & 80 papers across 3 drugs; digitized literature data; data not shared & Study-level meta-analysis of PK/PD relationships across SGLT2 inhibitors \\
\addlinespace[1pt]
Nemitz2025 \cite{Nemitz2025} & - & PBPK/PD & SBML, Python & \cellcolor{green!20}Yes & \cellcolor{green!20}Yes & \cellcolor{green!20}Yes & \cellcolor{green!20}Yes & \cellcolor{green!20}Yes & \cellcolor{green!20}Yes & \cellcolor{green!20}Yes & https://github.com/matthiaskoenig/dapagliflozin-model Model, Data, Code in github repository; CI/CD workflow for reproducibility; longterm storage on zenodo: https://doi.org/10.5281/zenodo.13987865  & 28 & 28 clinical studies; digitized literature data & Whole-body mechanistic PBPK/PD model of dapagliflozin linking ADME to SGLT2-mediated urinary glucose excretion and renal threshold for glucose, capturing dose dependency, renal and hepatic impairment, and food effects, implemented in SBML with full FAIR compliance \\
\addlinespace[1pt]
\bottomrule
\end{tabularx}

\end{threeparttable} 
\end{table}
\end{landscape}